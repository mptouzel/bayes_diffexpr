\subsection*{Identifying responding clones}

The posterior probability on expansion factors $\rho(s|n,n')$ (Eq.~\ref{eq:post}) can be used to
study the fate and dynamics of particular clones. For instance, we can
identify responding clones as having a low posterior probability of being not expanding $P_{\rm null}=\rho(s\leq 0|n,n')<0.025$. $P_{\rm null}$ is the Bayesian counterpart of a p-value but differs from it in a fundamental way: it gives the probability that expansion happened given the observations, when a p-value would give the probability of the observations in absence of expansion.  We can define a similar criterion for contracting clones.

To get the expansion or contraction factor of each clone, we can compute the posterior average and median, $\<s\>_{n,n'}=\int \textrm{d}s\,s \rho(s|n,n')$ and $s_{\rm median}$ ($F(s_{\rm median}|n,n')=0.5$, for the cumulative density function, $F(s|n,n')=\int_{-\infty}^s\rho(\tilde{s}|n,n')\textrm{d}\tilde{s}$),  corresponding to our best estimate for the log fold-change.
In \cref{fig:SM_smed_snaive}, we show how the median Bayesian estimator differs from the naive estimator $s_{\textrm{naive}}=\ln n^{\prime}/n$. While the two agree for large clones for which relative noise is smaller, the naive estimator over-estimates the magnitude of log fold-changes for small clones because of the noise. The Bayesian estimator accounts for that noise and gives a more conservative and more realistic estimate.

\begin{figure*}
\includegraphics[width=\textwidth]{fig8_volc_plus}
\centering{}
\caption{
\emph{Identifying responding clones}. (A) Plot of confidence of expanded response versus average effect size. A significance threshold is placed according to $P_{\textrm{null}}=0.025$, where $P_{\textrm{null}}=P(s\leq 0)$. (B) The same threshold for significant expansion in $(n,n^\prime)$-space with identified clones highlighted in red. (C) The optimal values of $\alpha$ and $\bar{s}$ for donor S2 and day-0 day-15 comparison for 3 replicates (square markers). The background heat map is the list overlap (the size of the intersection of the two lists divided by the size of their union) between a reference list obtained at the optimal $\hat\theta_{\rm exp}$ (black dot) and lists obtained at non-optimal $\theta_{\rm exp}$. (D) Mean posterior log fold-change $\<s\>_{\rho(s|n,n')}$ as a function of precursor frequency.
\label{fig:volcano}}
\end{figure*}

In \cref{fig:volcano}A, we show a `volcano' plot showing how both $P_{\rm null}$ and $\<s\>_{n,n'}$ vary as one scans values of the count pairs $(n,n')$. \Cref{fig:volcano}B shows the all count pairs $(n,n')$ between day $0$ and day $15$ following yellow fever vaccination, with red clones above the significance threshold line $P_{\rm null}=0.025$ being identified as responding.

Given the uncertainty in the expansion model parameters $\theta_{\rm exp}=(\bar s,\alpha)$, we wondered how robust our list of responding clonotypes was to those variations. In \cref{fig:volcano}C, we show the overlap of lists of strictly expanding clones ($P(s\leq 0|n,n')<0.025$) as a function of $\theta_{\rm exp}$, relative to the optimal value $\hat\theta_{\rm exp}$ (black circle). The ridge of high overlap values exactly mirrors the ridge of high likelihood values onto which the learned parameters fall (\cref{fig:diffexpr_ex2}). Values of $\hat \theta_{\rm exp}$ obtained for other replicate pairs (square symbols) fall onto the same ridge, meaning that these parameters lead to virtually identical lists of candidates for response.

The list of identified responding clones can be used to test hypotheses about the structure of the response. For example, recent work has highlighted a power law relationship between the initial clone size and clones subsequent fold change response in a particular experimental setting \cite{Mayer2019}. We can plot the relationship in our data as the posterior mean log fold change versus the posterior initial frequency, $f$ (\cref{fig:volcano}D). While the relationship is very noisy, emphasizing the diversity of the response, it is consistent with a decreasing dependency of the fold change with the clone size prior to the immune response.

The robustness of our candidate lists rests on their insensitivity to the details of how the model explains typical expansion. In \cref{fig:posteriors}, we show how the posterior belief varies significantly for count pairs $(0,n^\prime)$, $n^\prime>0$, across a range of values of $\bar{s}$ and $\alpha$ passing along the ridge of plausible models (\cref{fig:volcano}C). A transition from a low to high value of the most probable estimate for $s$ characterizes their shapes and arises as $\bar{s}$ becomes large enough that expansion from frequencies near $f_\textrm{min}$ is plausible, and the dominant mass of clones there makes this the dominate posterior belief. Thus, these posteriors are shaped by $\rho_s(s)$ at low $\bar{s}$, and $\rho(f)$ at high $\bar{s}$. Our lists vary negligibly over this transition, and thus are robust to it.

