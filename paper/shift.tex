\documentclass[letterpaper,english,prl,reprint,onecolumn]{revtex4-1} %twocolumn,

%\usepackage[latin9]{inputenc}
\setcounter{secnumdepth}{3}
\usepackage{babel}
\usepackage{amsmath}
\usepackage{amssymb}
\usepackage{graphicx} 
\usepackage{epstopdf} %add for pdflatex, nut don't compile because of invisible character copied from .bbl file?
\usepackage[T1]{fontenc}
\usepackage[utf8x]{inputenc} 
\usepackage{esint}
%\usepackage[hyphens]{url}
\usepackage[unicode=true]{hyperref}
\usepackage[table]{xcolor}
\usepackage{bbold}
% Additional Options
\begin{document}

\subsection*{Model Sampling}
According to Misha, the samples have roughly the same number of cells. Since the model involves expansion and contraction in the second condition, some normalization in the second condition is needed such that it produces roughly the same total number of cells as those in the first condition, consistent with the observed data. One approach (the one taken below) is to normalize at the level of clone frequencies. 

\subsubsection*{Direct Sampling}
The frequencies of the first condition, $f_i$, are sampled from $\rho(f)$ until they sum to 1 (i.e. until before they surpass 1, with a final frequency added that takes the sum exactly to 1). An equal number of log-fold changes, $s_i$, are sampled from $P(s)$. The normalized frequencies of the second condition are then $f'_i=f_ie^{s_i}/\sum_j f_je^{s_j}$.  Counts from the two conditions are then sampled from $P(n|f)$ and $P(n'|f')$, respectively. As a final step, unseen clones, i.e. those with $(n,n')=(0,0)$, are discarded.

\subsubsection*{Effective Sampling}
For an efficient implementation, the procedure should avoid sampling the numerous clones that produce $(n,n')=(0,0)$, since these are discarded. Such a procedure follows. 

First, the $(f,s)$-plane is partitioned into two regions, $D=\{(f,s)|f<f_0,fe^s<f_0\}$ and its complement, $\bar D$, with $f_0$ chosen such that clones sampled from $\bar D$ are often \textit{seen}, i.e. $n+n'>0$ (a minority of clones sampled in $\bar D$ will nevertheless give $n+n'=0$; these are discarded). In contrast, frequencies sampled from $D$ will be small, so that most will be unseen, and we must condition on the clone being seen when sampling from this regime. Moreover, their average is unaffected by the long-tailed behaviour of the distribution in the large-frequency regime and thus is well-approximated by the corresponding ensemble average. We use this latter fact when computing the renormalization of the frequencies of the second condition. 

We compute the mass in $D$ as $P_D=\int_f \textrm{d}f\rho(f)\sum_s P(s) \mathbb{1}((f,s)\in D)$.

We sample $(f,s)$ in $\bar D$ until the sum of the first condition's frequencies, $\sum_i f_i$ added to the expected sum in $D$, $P_D N_{cl}\langle f\rangle_{P(f|D)}$, equals 1,
\begin{align}
	1=\sum_{i=1}^{N_{\bar D}} f_i + P_D N_{cl}\langle f\rangle_{P(f|D)}\;,
\end{align}
where $N_{cl}$ is the total number of clones in the repertoire. The number sampled from $\bar D$, $N_{\bar D}$, is determined from that expression self-consistently by substituting $N_{cl}=N_{\bar D}/(1-P_D)$ obtained from $N_{\bar D}+P_D N_{cl}\equiv N_{cl}$. The normalization for the second condition's frequencies is then 
\begin{align}
	Z=\sum_{i=1}^{N_{\bar D}} f_ie^{s_i} + P_D N_{cl}\langle fe^s\rangle_{P(f,s|D)}
\end{align}
such that the second condition's frequencies are ${f_i'=f_ie^{s_i}/Z}$. Molecule counts are then sampled from $P(n|f)$ and $P(n'|f')$.  

We then sample from $D$ conditioned on the clone being seen, i.e. having produced a finite number of molecules in either of the two conditions. We thus sample $N_D=P(n+n'>0|D)P_D N_{cl}$ clones from $P(f,s|D,n+n'>0)$. To avoid having to sample over the joint distribution of $n$ and $n'$, we condition on the 3 regions of finite counts in both conditions, $(n,0)$, $(0,n')$, and $(n,n')$, in which $n$ and $n'$ can be sampled independently. 
Note the presence of the normalization factor, $Z$, in 
\begin{align}
	P(n+n'>0|D)=\int_f \textrm{d}f\rho(f)\sum_s P(s)(1-P(n=0|f)P(n'=0|f'=fe^s/Z))\;.
\end{align}
and
\begin{align}
	P(f,s|D,n+n'>0)=\frac{\rho(f)P(s)(1-P(n=0|f)P(n'=0|f'=fe^s/Z))}{P(n+n'>0|D)P_D}\;.
\end{align}
We then concatenate the $N_D$ sampled counts from $D$ and the $N_{\bar D}$ sampled counts (with $(n,n')=(0,0)$ realizations discarded) from $\bar D$ to obtain the full data set.  

% \subsection*{equal frequency constraint}
% 
% 
% To implement the constraint in the procedure to sample from the model that the mean frequencies over the repertoire is equal for both samples, we first sample from $P(f,s|n+n'>0)$ $N$ times producing $\{f_i\}_{i=1}^N$ and $\{s_i\}_{i=1}^N$, where $N$ is given and where we have chosen a shift, $s_0$ in order to compute $P(f,s|n+n'>0)$. Note that a choice of shift also fixes the observed fraction of the repertoire, $\alpha_{s_0}=1-\int\textrm{d}f\rho(f)P(n=0|f)\sum_s P_{s_0}(s)P(n'=0|f,s)$ and that for a given sample size of observed clones, $N$, the total number of clones is $M=N/\alpha_{s_0}$. (The maximum number of allowed clones, i.e. the number of lymphocytes in the body $m_T=10^{11}$, is already specified in $\rho(f)$, so we should ensure that $M<m_T$ for self-consistency.) Using the produced values, we compute the observed set of differentially expressed frequencies, $f_i'=f_i e^{s_i}$. We then compute the total average of the first and second frequencies, each of which we decompose in a seen and unseen contribution. The seen contribution we compute as an empirical average over the frequencies that we have sampled, and the unseen contribution we compute as an ensemble average. 
% \begin{align*}
% 	\bar{f}&=\alpha_{s_0} \frac{1}{N}\sum_{i=1}^N f_i+(1-\alpha_{s_0})\langle f\rangle_{P(f,s|n+n'=0)}\;,\\
% 	\bar {f'}&=\alpha_{s_0} \frac{1}{N}\sum_{i=1}^N f_i'+(1-\alpha_{s_0})\langle f'\rangle_{P(f,s|n+n'=0)}\;,
% \end{align*}
% where $f'=fe^s$. The constraint we wish to satisfy is $\bar {f'}=\bar {f}$, which we do by adjusting the shift, $s_0$ that appears in $\alpha_{s_0}$, $P(f,s|n+n'>0)$, and $P(f,s|n+n'=0)$. Namely, we set $s_0=\log\bar {f'}-\log\bar {f}$, and iterate until $s_0$ converges. Since we wish to then sample cells and molecules, we must implement the conditioning on $n_1+n_2$ by separating the domain of molecule counts in the four quadrants of zero/finite count value for each of pair of counts. We then sample the number of cell and molecules conditioned on the quadrant.
% 
% 
% % To implement the constraint in the procedure to sample from the model that the mean frequencies over the repertoire is equal for both samples, we first sample from $P(f,s|n+n'>0)$ $N$ times producing $\{f_i\}_{i=1}^N$ and $\{s_i\}_{i=1}^N$, where $N$ is given and where we have chosen a shift, $s_0$ in order to compute $P(f,s|n+n'>0)$. Note that a choice of shift also fixes the observed fraction of the repertoire, $\alpha_{s_0}=1-\int\textrm{d}f\rho(f)P(n=0|f)\sum_s P_{s_0}(s)P(n'=0|f,s)$ and that for a given sample size of observed clones, $N$, the total number of clones is $M=N/\alpha_{s_0}$. (The maximum number of allowed clones, i.e. the number of lymphocytes in the body $m_T=10^{11}$, is already specified in $\rho(f)$, so we should ensure that $M<m_T$ for self-consistency.) Using the produced values, we compute the observed set of differentially expressed frequencies, $f_i'=f_i e^{s_i}$. We then compute the total average of the first and second frequencies, each of which we decompose in a seen and unseen contribution. The seen contribution we compute as an empirical average over the frequencies that we have sampled, and the unseen contribution we compute as an ensemble average. 
% % \begin{align*}
% % 	\bar{f}&=\alpha_{s_0} \frac{1}{N}\sum_{i=1}^N f_i+(1-\alpha_{s_0})\langle f\rangle_{P(f,s|n+n'=0)}\;,\\
% % 	\bar {f'}&=\alpha_{s_0} \frac{1}{N}\sum_{i=1}^N f_i'+(1-\alpha_{s_0})\langle f'\rangle_{P(f,s|n+n'=0)}\;,
% % \end{align*}
% % where $f'=fe^s$. The constraint we wish to satisfy is $\bar {f'}=\bar {f}$, which we do by adjusting the shift, $s_0$ that appears in $\alpha_{s_0}$, $P(f,s|n+n'>0)$, and $P(f,s|n+n'=0)$. Namely, we set $s_0=\log\bar {f'}-\log\bar {f}$, and iterate until $s_0$ converges. Since we wish to then sample cells and molecules, we must implement the conditioning on $n_1+n_2$ by separating the domain of molecule counts in the four quadrants of zero/finite count value for each of pair of counts. We then sample the number of cell and molecules conditioned on the quadrant.
% 
% Alternatively, a brute force approach would be to sample a large number (order $M$) of the first clone frequencies, $\{f_i\}_{i=1}^M$ from $\rho(f)$, and the same number of log-fold changes from $P_{s_0}(s)$, $\{s_i\}_{i=1}^M$ from which the second frequencies are computed, $\{f'=f_ie^{s_i}\}_{i=1}^M$. Tn shoudl equset the shift to $s_0=\log(\sum_i f'_i/\sum_i f_i)$ such that $\sum_i f_i=\sum_i f'_i$. Then we sample cell counts and corresponding molecule counts for all of these clones, removing clones for which the sampling produced 0 molecules in both conditions.
% 
\subsection*{Inference}
%The null model is already learned, from which we compute $P(n,n'|s)$. 
The sampling procedure normalizes $N_{cl}\langle f \rangle$ and $N_{cl}\langle fe^s \rangle$. From sampled dataset, the first condition arises from expressing the average over $n$ and $n'$ and pulling out of the sum the unseen contribution, $n+n'=0$,
\begin{align*}
	\langle f \rangle =&P(0,0) \langle f\rangle_{P(f|0,0)}+\sum_{n+n'>0}P(n,n')\langle f\rangle_{P(f,s|n,n')} \;\\
	\langle f \rangle \approx&P(0,0) \langle f\rangle_{P(f|0,0)}+\sum_{(n,n')\in\mathcal{D}}\frac{\#(n,n')}{N_{cl}}\langle f\rangle_{P(f,s|n,n')} \;\\
	\langle f \rangle =&P(0,0) \langle f\rangle_{P(f|0,0)}+\frac{1}{N_{cl}}\sum_{i=1}^{N} \langle f\rangle_{P(f,s|n_i,n_i')} \;,
\end{align*}
where in the second line we have introduced the total number of clones in the repertoire to estimate the probability $P(n,n'))$ from the number of occurences of the pair in the dataset, $\mathcal{D}$. In the last line we resum over clones. The normalization condition is that the above expression should equal $1/N_{cl}$, the constraint thus reduces to 
\begin{align*}
	1=N_{cl}\langle f \rangle=P(0,0)N_{cl}\langle f\rangle_{P(f,s|0,0)}+\sum_{i=1}^{N} \langle f\rangle_{P(f,s|n_i,n_i')} \;,
\end{align*}
We estimate $N_{cl}$ using the identity, $N_{cl}\equiv N/(1-P(0,0))$ (which contains information about the normalization of the second frequency).
%Does ML satisfy this relation as well as the implicit relation from $N_{cl}$, $N= (1-P(0,0))/\langle f \rangle $ (lefthand side is from the data, righthand side is from the model)? Also, 

Does $\langle f \rangle$ above equal $\int_{f_{min}}^1 f\rho(f)\textrm{d}f$ and does it equal $1/N_{cl}$, where $N_{cl}\equiv N/(1-P(0,0))$? 

The second normalization obeys the same reasoning with the constraint  $\langle fe^s \rangle=1/N_{cl}$, 
\begin{align*}
	1 =\sum_{i=1}^{N} \langle fe^s\rangle_{P(f,s|n_i,n_i')} + P(0,0) N_{cl}\langle fe^s\rangle_{P(f,s|0,0)}\;.
\end{align*}


\newpage
\section{Full repertoire formulation}
For a sample dataset with $N_{samp}$ clones the generative model of the full repertoire is 
\begin{eqnarray}
	P(\vec{f},\vec{s},n,n')=\prod^{N_{samp}}_{i=1}P(n|f_i)P(n'|f_ie^{s_i}/Z) \rho(f_i)\rho(s_i)\rho(s_{N_{cl}})\prod^{N_{cl}-1}_{j=N_{samp}+1}\rho(f_j)\rho(s_j)\;,
\end{eqnarray}
where $\vec{f}=(f_1,\dots,f_{N_{cl}})$, $\vec{s}=(s_1,\dots,s_{N_{cl}})$, with the normalization constraint $f_{N_{cl}}=1-\sum_{i=1}^{N_{cl}-1} f_i$, and $Z=Z(\vec{f},\vec{s})=\sum^{N_{cl}}_{i=1}f_ie^{s_i}$ and the total number of clones, $N_{cl}$, must be determined self-consistently from $N_{cl}=N_{samp}/(1-P(0,0))$ where the marginal likelihood is
\begin{eqnarray}
	P(n,n')=\int\mathrm{d}\vec{f}_1^{N_{samp}}\mathrm{d}\vec{s}_1^{N_{samp}} \prod^{N_{samp}}_{i=1}P(n|f_i)\rho(f_i)\rho(s_i)\int\mathrm{d}\vec{f}_{N_{samp}+1}^{N_{cl}-1}\mathrm{d}\vec{s}_{N_{samp}+1}^{N_{cl}}P(n'|f_ie^{s_i}/Z)\rho(s_{N_{cl}})\prod^{N_{cl}-1}_{j=N_{samp}+1}\rho(f_j)\rho(s_j) \nonumber
\end{eqnarray}
with notation $\vec{f}_i^j=(f_i,\dots,f_j)$ and same for $s$. $\vec{f}_{N_{samp}+1}^{N_{cl}}$ and $\vec{s}_{N_{samp}+1}^{N_{cl}}$ appear only in $Z$. 

We can use EM to learn this model's parameters: those of all the $\rho(s)$. We average the log-likelihood weighted by the posterior of the hidden variables conditioned on the data, and we have assumed that the parameters learned on same-day data apply to all clones, seen or unseen. Denoting parameters by $\lambda$,
\begin{eqnarray}
 Q(\lambda|\lambda')=\sum_{(n,n')_{obs}}^{N_{samp}}\int\mathrm{d}\vec{f}_1^{N_{cl}-1}\mathrm{d}\vec{s}_1^{N_{cl}}\rho(\vec{f},\vec{s}|n,n',\lambda')\log \left[P(\vec{f},\vec{s},n,n'|\lambda)\right]\;.
\end{eqnarray}
Maximizing $Q$  with respect to $\lambda$ is made easier since $\lambda$ appears only in all the $\rho(s)$'s  which appear as factors in $P(\vec{f},\vec{s},n,n'|\lambda)$ and so all but these factor's vanish. The resulting expressions require integration involving the posterior,
\begin{align}
	\rho(\vec{f},\vec{s}|n,n',\lambda')=\frac{P(\vec{f},\vec{s},n,n'|\lambda')}{P(n,n')}\;.
\end{align}
For a simple case of $\rho(s)=\frac{\lambda}{2}e^{-\lambda|s|}$, performing the derivative of $Q$ and setting to zero gives,
\begin{eqnarray}
\sum_{(n,n')_{obs}}^{N_{samp}}\int\mathrm{d}\vec{f}_1^{N_{cl}-1}\mathrm{d}\vec{s}_1^{N_{cl}}\rho(\vec{f},\vec{s}|n,n',\lambda')\sum_{i=1}^{N_{cl}}\frac{1-|s_i|\lambda}{\lambda}=0\;.\\
\frac{N_{samp}N_{cl}}{\lambda}=\sum_{(n,n')_{obs}}^{N_{samp}}\sum_{i=1}^{N_{cl}}\int\mathrm{d}\vec{f}_1^{N_{cl}-1}\mathrm{d}\vec{s}_1^{N_{cl}}|s_i|\rho(\vec{f},\vec{s}|n,n',\lambda')\;.
\end{eqnarray}
Writing out each of these vector integrals and identifying the $j$th integrand,
\begin{align}
	&\sum_{(n,n')_{obs}}^{N_{samp}}\frac{1}{P(n,n')}\sum_{j=1}^{N_{cl}-1}\int\mathrm{d}\vec{f}_{1/j}^{N_{cl}-1}\mathrm{d}\vec{s}_{1/j}^{N_{cl}}\rho(s_{N_{cl}})\left(\prod_{k=1/j}^{N_{cl}-1}\rho(f_k)\rho(s_k)\right)\\
	&\left[\Theta(N_{samp}-j)\left(\prod_{i=1/j}^{N_{samp}}P(n|f_i)\right)\right.
		\int P(n|f_j)\rho(f_j)\mathrm{d}f_j\int \rho(s_j)|s_j|\left(P(n'|f_je^{s_j}/Z)\prod_{i=1/j}^{N_{samp}}P(n'|f_ie^{s_i}/Z)\right)\mathrm{d}s_j 
	 +\\
	&\left. \Theta(j-N_{samp}+1)\left(\prod_{i=1}^{N_{samp}}P(n|f_i)\right)
		\int \rho(f_j)\mathrm{d}f_j\int \rho(s_j)|s_j|\left(\prod_{i=1}^{N_{samp}}P(n'|f_ie^{s_i}/Z)\right)\mathrm{d}s_j\right]
\end{align}
where $/j$ denotes leaving out the $j$th index. Note that the integral expression in the first term includes a $P(n|f_j)$ factor, while those in the second term do not. Factorization of these integrals is prevented by $Z$, which contains all $s$ and $f$ variables. However, the many terms in the sum will decorrelate the resulting value so that the law of large numbers implies the sum should become nearly independent and converge to fixed value. In this case, the product in the integral in both terms can be brought out of the integral over $s_j$ and $f_j$. 
\begin{align}
	&\sum_{(n,n')_{obs}}^{N_{samp}}\frac{1}{P(n,n')}\sum_{j=1}^{N_{cl}-1}\int\mathrm{d}\vec{f}_{1/j}^{N_{cl}-1}\mathrm{d}\vec{s}_{1/j}^{N_{cl}}\rho(s_{N_{cl}})\left(\prod_{k=1/j}^{N_{cl}-1}\rho(f_k)\rho(s_k)\right)\\
	&\left[\Theta(N_{samp}-j)\left(\prod_{i=1/j}^{N_{samp}}P(n|f_i)P(n'|f_ie^{s_i}/Z)\right)\right.
		\int P(n|f_j)\rho(f_j)\mathrm{d}f_j\int \rho(s_j)|s_j|P(n'|f_je^{s_j}/Z)\mathrm{d}s_j 
	 +\\
	&\left. \Theta(j-N_{samp}+1)\left(\prod_{i=1}^{N_{samp}}P(n|f_i)P(n'|f_ie^{s_i}/Z)\right) \int \rho(s_j)|s_j|\mathrm{d}s_j\right]
\end{align}
In the first term, we see that this leaves an average of $|s|$ over the $j$th clone marginal, $P(n,n',f_j,s_j)$. In the second term, this leaves the average of $|s_j|$ over the prior, $\rho(s_j)$.
\begin{align}
	&\sum_{(n,n')_{obs}}^{N_{samp}}\frac{1}{P(n,n')}\sum_{j=1}^{N_{cl}-1}\int\mathrm{d}\vec{f}_{1/j}^{N_{cl}-1}\mathrm{d}\vec{s}_{1/j}^{N_{cl}}\rho(s_{N_{cl}})\left(\prod_{k=1/j}^{N_{cl}-1}\rho(f_k)\rho(s_k)\right)\\
	&\left[\Theta(N_{samp}-j)\left(\prod_{i=1/j}^{N_{samp}}P(n|f_i)P(n'|f_ie^{s_i}/Z)\right)\right.
		\langle|s_j|\rangle_{P(n,n',f_j,s_j)} 
	 +\\
	&\left. \Theta(j-N_{samp}+1)\left(\prod_{i=1}^{N_{samp}}P(n|f_i)P(n'|f_ie^{s_i}/Z)\right) \langle|s_j|\rangle_{\rho(s_j)}\right]
\end{align}
Note that $j$ appears in the second term only as a label. Distributing the multi-integral into the two terms, we see that $\vec{f}_{N_{samp}+1}^{N_{cl}}$ and $\vec{s}_{N_{samp}+1}^{N_{cl}}$ integrate out so what remains is only integrated over the observed sample. 
\begin{align}
	&\sum_{(n,n')_{obs}}^{N_{samp}}\frac{1}{P(n,n')}\sum_{j=1}^{N_{cl}-1}\left[\right.\\
	&\Theta(N_{samp}-j)\int\mathrm{d}\vec{f}_{1}^{N_{samp}}\mathrm{d}\vec{s}_{1}^{N_{samp}}\left(\prod_{i=1/j}^{N_{samp}}\rho(f_i)\rho(s_i)P(n|f_i)P(n'|f_ie^{s_i}/Z)\right)
		\langle|s_j|\rangle_{P(n,n',f_j,s_j)}
	 +\\
	&\left. \Theta(j-N_{samp}+1)\int\mathrm{d}\vec{f}_{1}^{N_{samp}}\mathrm{d}\vec{s}_{1}^{N_{samp}}\left(\prod_{i=1}^{N_{samp}}\rho(f_i)\rho(s_i)P(n|f_i)P(n'|f_ie^{s_i}/Z)\right) \langle|s_j|\rangle_{\rho(s_j)}\right]
\end{align}
Now, the integral in the first term is similar to the marginal, $P(n,n')$, but lacks the $j$th clone so we denote it $P_{/j}(n,n')$. The integral in the second term is the marginal (in this approximation where $Z$ is fixed) and so cancels. 
\begin{align}
	&\sum_{(n,n')_{obs}}^{N_{samp}}\sum_{j=1}^{N_{cl}-1}\left[\Theta(N_{samp}-j)\frac{P_{/j}(n,n')}{P(n,n')}\langle|s_j|\rangle_{P(n,n',f_j,s_j)}+\Theta(j-N_{samp}+1)\langle|s_j|\rangle_{\rho(s_j)}\right]
\end{align}
We now can distribute the sum over $j$. Since the priors are all the same, we remove the $j$-dependence on the second term and the sum distributes over the second term simply as a factor equal to the number of clones. Switching the order for clarity,
\begin{align}
	&\sum_{(n,n')_{obs}}^{N_{samp}}\left[(N_{cl}-N_{samp})\langle|s|\rangle_{\rho(s)}+\sum_{j=1}^{N_{samp}}\frac{P_{/j}(n,n')}{P(n,n')}\langle|s_j|\rangle_{P(n,n',f_j,s_j)}\right]
\end{align}
Finally, we note that the fraction $\frac{P_{/j}(n,n')}{P(n,n')}$ leaves the $j$th marginal, $P_j(n,n')$ in the denominator, which combines with $P(n,n',f_j,s_j)$ to make the average over the $j$th posterior, $P(s_j,f_j|n,n')$,
\begin{align}
	&\sum_{(n,n')_{obs}}^{N_{samp}}\left[(N_{cl}-N_{samp})\langle|s|\rangle_{\rho(s)}+\sum_{j=1}^{N_{samp}}\langle|s_j|\rangle_{P(f_j,s_j|n,n')}\right]\;.
\end{align}
The same argument regarding the independence of $j$ now applies to the second (previously first) term,
\begin{align}
	&\sum_{(n,n')_{obs}}^{N_{samp}}\left[(N_{cl}-N_{samp})\langle|s|\rangle_{\rho(s)}+N_{samp}\langle|s|\rangle_{P(f,s|n,n')}\right]\;.
\end{align}
Putting this back into our EM solution equation,
\begin{align}
	\frac{N_{samp}N_{cl}}{\lambda}=\sum_{(n,n')_{obs}}^{N_{samp}}\left[(N_{cl}-N_{samp})\langle|s|\rangle_{\rho(s)}+N_{samp}\langle|s|\rangle_{P(f,s|n,n')}\right]\;.\\
	\frac{1}{\lambda}=\frac{1}{N_{samp}}\sum_{(n,n')_{obs}}^{N_{samp}}\left[\left(1-\frac{N_{samp}}{N_{cl}}\right)\langle|s|\rangle_{\rho(s)}+\frac{N_{samp}}{N_{cl}}\langle|s|\rangle_{P(f,s|n,n')}\right]\\
	\frac{1}{\lambda}=\frac{1}{N_{samp}}\sum_{(n,n')_{obs}}^{N_{samp}}\left[P(n+n'=0)\langle|s|\rangle_{\rho(s)}+P(n+n'>0)\langle|s|\rangle_{P(f,s|n,n')}\right]\;,
\end{align}
where we have used the definition of $N_{cl}=N_{samp}/(1-P(n+n'=0))$. The hidden and visible part of the repertoire contribute a weighted prior and posterior average respectively. The stable point of iterating this procedure occurs when the prior parameter, $\bar{s}$, equals the result, so that prior average equals posterior average.
\begin{align}
	\frac{1}{\lambda}=\frac{1}{N_{samp}}\sum_{(n,n')_{obs}}^{N_{samp}}\left[P(n+n'=0)\frac{1}{\lambda}+(1-P(n+n'=0))\langle|s|\rangle_{P(f,s|n,n')}\right]\\
	\frac{1}{\lambda}=\frac{1}{N_{samp}}\sum_{(n,n')_{obs}}^{N_{samp}}\langle|s|\rangle_{P(f,s|n,n')}\;.
\end{align}
% sampled dataset, these averages are, en inferring the $P(s)$ parameters from a dataset of pairs of molecule counts (such as the one obtained from the above sampling procedure), we should incorporate these normalizations. Normalizing the The average total number of clones, $N_{cl}=\langle f \rangle^{-1}$, where $\langle f \rangle=\int_{f_{min}}^1f\rho(f)\textrm{d}f=\frac{\alpha}{\alpha-1}f_{min}$. Nevertheless, $N_{cl}P(0,0)+N=N_{cl}$ should also hold, so that $\langle f \rangle=(1-P(0,0))\frac{1}{N}$. Does ML satisfy this latter relation? The other normalization If so, we set the normalization using the constraint on the data averaged posteriors of the frequencies, $\langle f \rangle_{data}=\langle f' \rangle_{data}$, where
% \begin{align}
% 	\bar{f}_{data}=\alpha_{s_0}\frac{1}{N}\sum_{i=1}^N \langle f \rangle_{P_{s_0}(f,s|n_i,{n'}_i)}+(1-\alpha_{s_0})\langle f\rangle_{P_{s_0}(f,s|0,0)}\\
% \end{align}
% so that $s_0=\log\langle f' \rangle_{data}-\log\langle f \rangle_{data}$. In practise, we precompute $P(n,n'|s)$.



% \subsection*{equal frequency constraint}
% The constraint of equal frequencies in the two compared conditions, $\langle f_1\rangle=\langle f_2\rangle$ can be satisfied with a suitable choice of the shift parameter, $s_0$, in the prior for differential expression, 
% \begin{align}
% 	P_{s_0}(s)=\frac{\alpha \beta}{Z_+}e^{-\frac{|s-s_0|}{s_+}}\Theta(s-s_0) +\frac{\alpha (1-\beta)}{Z_-}e^{-\frac{|s-s_0|}{s_-}}\Theta(s_0-s)+(1-\alpha)\delta(s-s_0)
% \end{align}
% The ensemble average can be evaluated over $P(f,s|n_1+n_2>0)$, where $f_1=f$ and $f_2=f e^s$, where 
% \begin{align}
% 	P(f,s|n_1+n_2>0)&=\frac{\sum_{n_1+n_2>0}{P(n_1,n_2,f,s)}}{\sum_{n_1+n_2>0}\int{\textrm{d}f \sum_s P(n_1,n_2,f,s)}} \\
% 					&=\frac{P(n_1+n_2>0,f,s)}{P(n_1+n_2>0)}
% \end{align}
% and using this, $P(f|n_1+n_2>0)=\sum_s P(f,s|n_1+n_2>0)$. The shift enters in $P(n_1,n_2,f,s)=P(n_1|f)P(n_2|f,s)\rho(f)P_{s_0}(s)$ via $P_{s_0}(s)$. A convenient change of variables $s\leftarrow\Delta s+s_0$ maps $P_{s_0}(s)$ to $P_{0}(\Delta s)$, upon which
% \begin{align}
% 	\langle fe^s \rangle &=\int{ \textrm{d}f \sum_{\Delta s=s_{min}-s_0}^{s_{max}-s_0} fe^{\Delta s+s_0}
% 	\frac{\sum_{n_1+n_2>0} P(n_1|f)P(n_2|f,\Delta s+s_0)\rho(f)P_{0}(\Delta s)}{P(n_1+n_2>0)}} \\
% 						 &=e^{s_0}\int{ \textrm{d}f \sum_{\Delta s=s_{min}-s_0}^{s_{max}-s_0} fe^{\Delta s} \frac{\sum_{n_1+n_2>0} P(n_1|f)P(n_2|f,\Delta s+s_0)\rho(f)P_{0}(\Delta s)}{P(n_1+n_2>0)}}
% \end{align}
% denoting the remaining integral, $\tilde{\langle fe^s \rangle}$, and performing the same change of variables on $\langle f \rangle$,
% \begin{align}
% 	\langle f \rangle &=\int{ \textrm{d}f \sum_{\Delta s=s_{min}-s_0}^{s_{max}-s_0} f            \frac{\sum_{n_1+n_2>0}  P(n_1|f)P(n_2|f,\Delta s+s_0)\rho(f)P_{0}(\Delta s)}{P(n_1+n_2>0)}}\;,
% \end{align}
% and so the condition can be written as $s_0=\ln \tilde{\langle fe^s \rangle} - \ln \langle f \rangle$. To obtain $s_0$ from this implicit equation, we apply an iterative scheme beginning with $s_0=0$. We compute $P(n_2|f,\Delta s+s_0)$, and then the latter expression supplies $s_0$ in the next iteration. In practice, we take a bounded range of $\Delta s$ symmetric around 0. Thus, the only factor containing shift information is $P(n_2|f,\Delta s+s_0)$ appearing in both $\tilde{\langle fe^s \rangle}$ and $\langle f \rangle$. However, for correspondence with numerics, the $e^{\Delta s}$ factor must be defined over a shifted domain, a fact that I can't immediately explain.
% 
% Alternatively, the average can be computed over the data directly, The should converge to the result of the above method in the limit of many clones (and some other condition?).


\end{document}