High-throughput immune repertoire sequencing (RepSeq) experiments are becoming a common way to study the diversity, structure and composition of lymphocyte repertoires, promising to yield unique insight into individuals' past infection history. However, the analysis of these sequences remains challenging, especially when comparing two different temporal or tissue samples. Here we develop a new theoretical approach and methodology to extract the characteristics of the lymphocyte repertoire response from different samples. The method is specifically tailored to RepSeq experiments and accounts for the multiple sources of noise present in these experiments. Its output provides expansion parameters, as well as a list of potentially responding clonotypes. We apply the method to describe the response to yellow fever vaccine obtained from samples taken at different time points. We also use our results to estimate the diversity and clone size statistics from data.
