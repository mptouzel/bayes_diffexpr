High-throughput sequencing of B- and T-cell receptors makes it possible to track immune repertoires across time, in different tissues, and in acute and chronic diseases or in healthy individuals. However, quantitative comparison between repertoires is confounded by variability in the read count of each receptor clonotype due to sampling, library preparation, and expression noise. Here, we present a general Bayesian approach to disentangle repertoire variations from these stochastic effects. Using replicate experiments, we first show how to learn the natural variability of read counts by inferring the distributions of clone sizes as well as an explicit noise model relating true frequencies of clones to their read count. We then use that null model as a baseline to infer a model of clonal expansion from two repertoire time points taken before and after an immune challenge. Applying our approach to yellow fever vaccination as a model of acute infection in humans, we identify candidate clones participating in the response.
